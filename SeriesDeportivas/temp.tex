\documentclass{article}
\usepackage{graphicx}
\usepackage[spanish]{babel}
\usepackage{float}

\begin{document}

\title{Predicción de series deportivas Investigación De Operaciones  (2 semestre - 2016)}
\author{Leonardo Mendoza - Patrick Maynard}

\maketitle

\begin{abstract}
Este algoritmo permite saber cuales son las posibilidades de ganar en una serie de juegos. Tomando en cuenta que se tiene la probabilidad de ganar en casa y como visitante. 
\end{abstract}

\section{Datos iniciales}

Estos son los datos iniciales son

\centering 
\begin{figure}[H]
\label{my-label2}
\begin{tabular}{|l|l|} 
\hline
 Cantidad de partidos & 3 \\ \hline
 Probabilidad de ganar en casa & 52 \\ \hline
 Probabilidad de ganar de visita & 48 \\ \hline
 Serie & 2,3,2 \\ \hline

\end{tabular}
\end{figure}
    
\section{Tabla de probabilidades}


\centering 
\begin{figure}[H]
\label{my-label2}
\begin{tabular}{|l|l|l|l|}
\hline
 1,000000& 1,000000& 1,000000& 1,000000\\ \hline
0,000000& 0,480000& 0,710400& 0,820992\\ \hline
0,000000& 0,230400& 0,451584& 0,661739\\ \hline
0,000000& 0,110592& 0,292331& 0,496117\\ \hline
\end{tabular}
\end{figure}


\section{Conclusion}
Usando la tabla anterior se puede saber cuales son las probabilidades de ganar si me faltan i juegos y al equipo contrario j juegos para ganar. 
\end{document}